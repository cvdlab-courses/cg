\documentclass[ignorenonframetext,]{beamer}
\usepackage{amssymb,amsmath}
\usepackage{ifxetex,ifluatex}
\usepackage{fixltx2e} % provides \textsubscript
\ifxetex
  \usepackage{fontspec,xltxtra,xunicode}
  \defaultfontfeatures{Mapping=tex-text,Scale=MatchLowercase}
\else
  \ifluatex
    \usepackage{fontspec}
    \defaultfontfeatures{Mapping=tex-text,Scale=MatchLowercase}
  \else
    \usepackage[utf8]{inputenc}
  \fi
\fi
\usepackage{listings}
\usepackage{url}

% Comment these out if you don't want a slide with just the
% part/section/subsection/subsubsection title:
\AtBeginPart{
  \let\insertpartnumber\relax
  \let\partname\relax
  \frame{\partpage}
}
\AtBeginSection{
  \let\insertsectionnumber\relax
  \let\sectionname\relax
  \frame{\sectionpage}
}
\AtBeginSubsection{
  \let\insertsubsectionnumber\relax
  \let\subsectionname\relax
  \frame{\subsectionpage}
}

\setlength{\parindent}{0pt}
\setlength{\parskip}{6pt plus 2pt minus 1pt}
\setlength{\emergencystretch}{3em}  % prevent overfull lines
\setcounter{secnumdepth}{0}
%----macros begin-----------------------------------------------------------------------------------
\usepackage{graphicx}
\usepackage{color}
\usepackage{xcolor}
\usepackage{amsthm}
\usepackage{hyperref}
\newcommand{\Alert}[1]{{\color{cyan}#1}}


%\renewenvironment{Shaded}{\pause\begin{snugshade}}{\end{snugshade}}
\def\twocolumns#1#2{\begin{columns}
\begin{column}{0.5\linewidth}#1\end{column}
\begin{column}{0.5\linewidth}#2\end{column}
\end{columns}}
\def\mytwocolumns#1#2#3#4{\begin{columns}
\begin{column}{#1\linewidth}#2\end{column}
\begin{column}{#3\linewidth}#4\end{column}
\end{columns}}
\def\mythreecolumns#1#2#3#4#5#6{\begin{columns}
\begin{column}{#1\linewidth}#2\end{column}
\begin{column}{#3\linewidth}#4\end{column}
\begin{column}{#5\linewidth}#6\end{column}
\end{columns}}
\def\threecolumns#1#2#3{\begin{columns}
\begin{column}{0.33\linewidth}#1\end{column}
\begin{column}{0.33\linewidth}#2\end{column}
\begin{column}{0.33\linewidth}#3\end{column}
\end{columns}}
\def\fourcolumns#1#2#3#4{\begin{columns}%
\begin{column}{0.25\linewidth}#1\end{column}%
\begin{column}{0.25\linewidth}#2\end{column}%
\begin{column}{0.25\linewidth}#3\end{column}%
\begin{column}{0.25\linewidth}#4\end{column}%
\end{columns}}

\def\textbf#1{\alert{#1}}
\def\emph#1{{\color{cyan}#1}}
\def\conv{\mbox{\textrm{conv}\,}}
\def\aff{\mbox{\textrm{aff}\,}}
\def\E{\mathbb{E}}
\def\R{\mathbb{R}}
\def\Z{\mathbb{Z}}
\def\N{\mathbb{N}}
\def\v#1{{\bf #1}}
\def\p#1{{\bf #1}}
\def\T#1{{\bf #1}}
\def\vet#1{{\left(\begin{array}{cccccccccccccccccccc}#1\end{array}\right)}}
\def\mat#1{{\left(\begin{array}{cccccccccccccccccccc}#1\end{array}\right)}}

\def\lin{\mbox{\rm lin}\,}
\def\aff{\mbox{\rm aff}\,}
\def\pos{\mbox{\rm pos}\,}
\def\cone{\mbox{\rm cone}\,}
\def\conv{\mbox{\rm conv}\,}
\newcommand{\homog}[0]{\mbox{\rm homog}\,}
\newcommand{\relint}[0]{\mbox{\rm relint}\,}

%------------------------------------------------------------
%------------------------------------------------------------
\usepackage{color}
\usepackage{listings}
\usepackage{textcomp}
\usepackage{setspace}
%\usepackage{palatino}
\renewcommand{\lstlistlistingname}{python}
\renewcommand{\lstlistingname}{python}
\definecolor{gray}{gray}{0.5}
\definecolor{green}{rgb}{0,0.5,0}
\definecolor{myblue}{rgb}{0.6,0.8,1}
%-------------------------------------------------------------------------------------
\definecolor{lbcolor}{rgb}{0.97,0.97,0.97}
\lstnewenvironment{python}[1][]{
\lstset{
language=python,
basicstyle=\ttfamily\scriptsize\setstretch{1},
stringstyle=\color{red},
showstringspaces=false,
alsoletter={1234567890},
otherkeywords={\ , \}, \{},
keywordstyle=\color{blue},
emph={access,and,break,class,continue,def,del,elif ,else,%
except,exec,finally,for,from,global,if,import,in,i s,%
lambda,not,or,pass,print,raise,return,try,while},
emphstyle=\color{black}\bfseries,
emph={[2]True, False, None, self},
emphstyle=[2]\color{green},
emph={[3]from, import, as},
emphstyle=[3]\color{blue},
upquote=true,
morecomment=[s]{"""}{"""},
commentstyle=\color{green}\slshape,
%emph={[4]1, 2, 3, 4, 5, 6, 7, 8, 9, 0},
emphstyle=[4]\color{blue},
literate=*{:}{{\textcolor{blue}:}}{1}%
{=}{{\textcolor{blue}=}}{1}%
{-}{{\textcolor{blue}-}}{1}%
{+}{{\textcolor{blue}+}}{1}%
{*}{{\textcolor{blue}*}}{1}%
{!}{{\textcolor{blue}!}}{1}%
{(}{{\textcolor{blue}(}}{1}%
{)}{{\textcolor{blue})}}{1}%
{[}{{\textcolor{blue}[}}{1}%
{]}{{\textcolor{blue}]}}{1}%
{<}{{\textcolor{blue}<}}{1}%
{>}{{\textcolor{blue}>}}{1},%
backgroundcolor=\color{lbcolor},rulecolor=,
frame=tb,
framexleftmargin=1mm, framextopmargin=1mm, rulesepcolor=\color{blue},#1
}}{}
%-------------------------------------------------------------------------------------
\beamerdefaultoverlayspecification{<+->}
%----macros end-----------------------------------------------------------------------------------

\title{Introduction to Python and pyPLaSM}
\author{Computational Visual Design Laboratory (\url{https://github.com/cvlab})
                ``Roma Tre'' University, Italy}
\date{Computational Graphics -- Lecture 5 -- March 11, 2013}

\begin{document}
\frame{\titlepage}

\begin{frame}

\tableofcontents

\end{frame}

\section{Starting Python}

\begin{frame}\frametitle{Install Python, Scipy, and pyOpenGL}

\begin{itemize}
\item
  \href{http://python.org/about/}{\emph{About Python}}
\item
  \href{http://scipy-lectures.github.com/}{\emph{Python Scientific
  Lecture Notes}}
\item
  \href{http://docs.python.org/2/tutorial/}{\emph{The Python Tutorial}}
\item
  \href{http://pyopengl.sourceforge.net/}{\emph{PyOpenGL: The Python
  OpenGL Binding}}
\item
  \href{http://jakevdp.github.com/blog/2012/09/20/why-python-is-the-last/}{\emph{Why
  Python Is the Last Language You'll Have to Learn}}
\end{itemize}

\end{frame}

\begin{frame}\frametitle{Install IPython as your IDE}

\begin{itemize}
\itemsep1pt\parskip0pt\parsep0pt
\item
  \href{http://ipython.org/}{\emph{The official IPython site}}
\end{itemize}

\vfill

\begin{itemize}
\itemsep1pt\parskip0pt\parsep0pt
\item
  \href{http://ipython.org/ipython-doc/stable/interactive/tutorial.html}{\emph{Introducing
  IPython}}
\end{itemize}

\end{frame}

\begin{frame}[fragile]\frametitle{Getting started}

\begin{python}
paoluzzi$ ipython
Python 2.7.2 (default, Jun 20 2012, 16:23:33) 
Type "copyright", "credits" or "license" for more information.

IPython 0.14.dev -- An enhanced Interactive Python.
?         -> Introduction and overview of IPython's features.
%quickref -> Quick reference.
help      -> Python's own help system.
object?   -> Details about 'object', use 'object??' for extra details.

In [1]: 
\end{python}

\end{frame}

\section{Geometric Programming}

\begin{frame}\frametitle{The design language PLaSM}

The design language PLaSM is a geometry-oriented extension of a subset
of FL.

\small

\begin{block}{FL Language}

\href{http://en.wikipedia.org/wiki/FL_programming_language}{\emph{FL
(programming at Function Level)}} is a language developed by the
Functional Programming Group of IBM Research Division at Almaden (USA)
{[}@BWW90, @BWWLA89{]}. The FL language, on the line of the Backus'
Turing lecture {[}Backus78{]} introduces an algebra over programs and
has an awesome expressive power.

\end{block}

\begin{block}{PLaSM Language}

\href{http://plasm.net/}{\emph{PLaSM, (the Programming LAnguage for
Solid Modeling)}} is a ``design language'' for geometric and solid
parametric design, developed by the CAD Group at the Universities ``La
Sapienza'' and ``Roma Tre'' {[}PS92, PPV95{]}. The language is strongly
inFLuenced by FL. With few sintactical differences, it can be considered
a geometric extension of a FL subset.

\end{block}

\end{frame}

\begin{frame}\frametitle{PLaSM Language}

\begin{columns}
\begin{column}{5cm}

\begin{thebibliography}{}

\bibitem[Paoluzzi {\em et~al.}, 1995]{plasm}
Paoluzzi, A., Pascucci, V.  \& Vicentino, M. (1995{\em{}}).
\newblock \href{http://dl.acm.org/citation.cfm?id=212349}{Geometric programming: a programming approach to geometric design}.
\newblock {\em {ACM} Trans. Graph.} {\bf 14} (3), 266--306.

\end{thebibliography}

\end{column}
\begin{column}{5cm}
   \centering
   \href{http://onlinelibrary.wiley.com/book/10.1002/0470013885}{\includegraphics[width=0.8\linewidth]{Paoluzzi}}
\end{column}
\end{columns}

\end{frame}

\begin{frame}\frametitle{Motivations for a Python port of PLaSM}

\begin{itemize}[<+->]
  \vfill \item Python: \Alert{multi-paradigm language} with efficient built-in data structures and simple/effective approach to OO programming.    
       
  \vfill \item Python's elegant syntax and dynamic typing, and its interpreted nature, make it ideal for \Alert{scripting} and \Alert{RAD}
      
  \vfill \item We wished for easy access to \Alert{Biopython}, \Alert{NumPy}, \Alert{SciPy}, \Alert{Femhub}, and the geometry libraries already interfaced with Python      
  \end{itemize}

\begin{alertblock}<+->{The easiest solution?}
   \Alert{Pyplasm}: Plasm $\rightarrow$ Python
\end{alertblock}

\end{frame}

\begin{frame}[fragile]\frametitle{First pyplasm tests}

generate and view a geometric object (hpc type) in pyplasm

\begin{python}
In [1]: from pyplasm import *
Evaluating fenvs.py..
...fenvs.py imported in 0.006975 seconds

In [2]: VIEW(CUBOID([1,4,9]))
\end{python}

\end{frame}

\begin{frame}[fragile]\frametitle{First pyplasm tests}

\begin{python}
from pyplasm import *
VIEW(CUBOID([1,4,9]))
VIEW(COLOR(BLACK)(CUBOID([1,4,9])))
\end{python}

\lstinline!COLOR! is a \textbf{second order function}: needs TWO
applications

\end{frame}

\begin{frame}[fragile]\frametitle{First pyplasm tests}

\begin{python}
a = [[0,0],[4,2],[2.5,3],
 [4,5],[2,5],[0,3],
 [-3,3],[0,0]]
VIEW(POLYLINE(a))
\end{python}

\end{frame}

\begin{frame}[fragile]\frametitle{First pyplasm tests}

\begin{python}
b = [[0,3],[0,1],[2,2],
 [2,4],[0,3]]
c = [[2,2],[1,3],[1,2],
 [2,2]]
AA(POLYLINE)([a,b,c])
VIEW(STRUCT(AA(POLYLINE)([a,b,c])))

polylines = AA(POLYLINE)([a,b,c])
polygon = SOLIDIFY(STRUCT(polylines))
VIEW(polygon)

cells = SKELETON(1)(polygon)
VIEW(cells)

solid = PROD([polygon, Q(0.5)])
VIEW(solid)

solid = PROD([polygon, QUOTE([0.5,-2.5,0.5])])
VIEW(solid)

complement = DIFFERENCE([ BOX([1,2,3])(solid), solid ])
VIEW(complement)
\end{python}

\end{frame}

\begin{frame}\frametitle{Assignments}

\begin{itemize}
\itemsep1pt\parskip0pt\parsep0pt
\item
  install python
\item
  install scipy
\item
  install pyplasm
\item
  explore \href{http://docs.python.org/2/tutorial/}{\emph{The Python
  Tutorial}}
\end{itemize}

\end{frame}

\end{document}
